\documentclass[a4paper,11pt]{article}

\usepackage[T1]{fontenc}
\usepackage[utf8]{inputenc}
\usepackage{graphicx}
\usepackage{xcolor}

\renewcommand\familydefault{\sfdefault}
\usepackage{tgheros}
\usepackage[defaultmono]{droidmono}

\usepackage{amsmath,amssymb,amsthm,textcomp}
\usepackage{enumerate}
\usepackage{multicol}
\usepackage{tikz}

\usepackage{geometry}
\geometry{left=25mm,right=25mm,%
bindingoffset=0mm, top=20mm,bottom=20mm}


\linespread{1.3}

\newcommand{\linia}{\rule{\linewidth}{0.5pt}}

% custom theorems if needed
\newtheoremstyle{mytheor}
    {1ex}{1ex}{\normalfont}{0pt}{\scshape}{.}{1ex}
    {{\thmname{#1 }}{\thmnumber{#2}}{\thmnote{ (#3)}}}

\theoremstyle{mytheor}
\newtheorem{defi}{Definition}

% my own titles
\makeatletter
\renewcommand{\maketitle}{
\begin{center}
\vspace{2ex}
{\huge \textsc{\@title}}
\vspace{1ex}
\\
\linia\\
\@author \hfill \@date
\vspace{4ex}
\end{center}
}
\makeatother
%%%

% custom footers and headers
\usepackage{fancyhdr}
\pagestyle{fancy}
\lhead{}
\chead{}
\rhead{}
\cfoot{}
\rfoot{Página \thepage}
\renewcommand{\headrulewidth}{0pt}
\renewcommand{\footrulewidth}{0pt}
%

% code listing settings
\usepackage{listings}
\lstset{
    language=C++,
    basicstyle=\ttfamily\small,
    aboveskip={1.0\baselineskip},
    belowskip={1.0\baselineskip},
    columns=fixed,
    extendedchars=true,
    breaklines=true,
    tabsize=4,
    prebreak=\raisebox{0ex}[0ex][0ex]{\ensuremath{\hookleftarrow}},
    frame=lines,
    showtabs=false,
    showspaces=false,
    showstringspaces=false,
    keywordstyle=\color[rgb]{0.627,0.126,0.941},
    commentstyle=\color[rgb]{0.133,0.545,0.133},
    stringstyle=\color[rgb]{01,0,0},
    numbers=left,
    numberstyle=\small,
    stepnumber=1,
    numbersep=10pt,
    captionpos=t,
    escapeinside={\%*}{*)}
}

%%%----------%%%----------%%%----------%%%----------%%%

\begin{document}

\title{Relatório do ep0 - mac323}

\author{Davi de Menezes Pereira, 11221988}

\date{16/03/2020}

\maketitle

\section*{Como foi feito o Exercício Programa}

Breve explicação de como eu entendi o enunciado do ep e como fiz o meu programa (sem muitos detalhes). \newline

Primeiramente, para cada instante t de tempo, o programa lê um número K, o número de aviões entrando em contato naquele momento. Depois disso, ele envia cada avião para sua respectiva fila.\newline

Foram feitas duas filas principais, uma para aviões de emergência e outra para aviões "normais" (que não são de emergência) e na classe dos aviões foi deixado um bool para dizer se o avião é de emergência ou não. Ou seja, não há especificação sobre qual é o tipo da emergência (isto vale para todo o programa: as emergências foram tratadas igualmente porque isso não estava muito claro no enunciado).\newline

Depois disso, temos que mandar os aviões paras as pistas, que são representadas no código por três variáveis do tipo "Plane". 
Para fazer isso, eu checo quanto tempo o avião "está" na pista (considerando que cada avião gasta 3 tempos para fazer uma ação) e, além disso, checo o tempo estimado do avião.
Esse útlimo passo foi uma alternativa encontrada para evitar que o avião seja enviado mais de uma vez pra fila dos que pousaram ou decolaram (para cada avião que já foi mandado pra lista de pouso ou decolagem, eu coloco um valor negativo no tempo estimado). Caso a pista esteja livre, eu coloco primeiro os aviões de emergência, e, se não tiver mais aviões de emergência, coloco algum da fila normal (nesse passo há pequenas alterações para o caso da pista 3). Para maior eficiência na alocação dos aviões, essa etapa é feita primeiramente para a pista 3.\newline

Sobre a eficiência do programa: No enunciado do ep foi pedido pra fazer o gerenciamento das pista de forma eficiente, entretando devido a algumas restrições do enunciado e alguns detalhes não terem ficado muito claros, optei por fazer dessa forma explicada acima (com duas filas principais). Não acho que é a forma mais eficiente, porém se for considerado o número de aviões que devem chegar para 3 pistas, essa maior eficiência não iria ter um impacto tão grande no tempo gasto pelo programa. Além disso, optei por não criar mais classes, porque vi que dava para trabalhar com o que tinha já. (Por exemplo, poderia ter criado uma classe para as pistas, mas preferi usar a classe avião que já havia sido criada).\newline

Sobre a organização do programa: foram feitas classes para os aviões (na verdade é para representar um voo) e uma estrutura de fila, embora essa estrutura não seja a tradicional, uma vez que foram necessárias algumas funções auxiliares que não existem numa fila tradicional. Ainda sobre as filas, optei por não usar template, considerando que a fila só foi usada para um tipo conhecido. Sobre a classe dos aviões, optei por deixar os atributos como público. \newline

\section*{Formato do input}

O formato considerado para input é de acordo com classe "Plane" (mostrado na figura abaixo).\newline

\begin{lstlisting}[label={list:first},caption= Parte do código em plane.h.]
    class Plane
    {
        public:
            string id;
            string from_to;
    
            bool emergency;
    
            int estimated_time;
            int waiting_time;
    
        public:
            Plane(); /* default constructor */
            Plane(string t_id, string t_from_to, bool t_landing, bool emergency, int t_fuel, int t_estimated_time);
            int track_time;
            int fuel;
            bool landing;
    };
\end{lstlisting}


O código abaixo indica como estão sendo lidos os aviões. As parte que não são lidas é porque já tem valor definido (como wating time que inicia com 0 e track time que inicia com 3) \newline


\begin{lstlisting}[label={list:first},caption= Parte do código em airport.cpp que lê os aviões (C++).]
cin >> K;
for(k = 0; k < K; k++)
{
    cin >> t_id;
    cin >> t_from_to;

    cin >> t_landing;
    cin >> t_emergency;

    cin >> t_fuel;
    cin >> t_estimated_time;

    Plane aux {t_id, t_from_to, t_landing, t_emergency, t_fuel, t_estimated_time};
        
    if(t_emergency) emergencies.push(aux);
    else normal.push(aux);
}
\end{lstlisting}


\section*{Formato da saída} 

A saída está sendo impressa com um formato específico em 7 partes diferentes  para cada instante de tempo (abaixo está a parte do código de impressão das filas), além de algumas outras infromações que podem ser vistas nos exemplos.\newline

1) Imprime as 5 diferentes filas de aviões. Cada fila é impressa de acordo com o código da figura abaixo. \newline

2) Tempo de espera dos aviões que querem pousar (no enunciado está dizendo para imprimir a média, mas optei por imprimir o número total)\newline

3) Tempo de espera dos aviões que querem decolar (no enunciado está dizendo para imprimir a média, mas optei por imprimir o número total)\newline

4) Quantidade de combustível dos aviões em espera. \newline

5) Quantidade de combustível dos aviões que pousaram.\newline

6) Quantidade total de aviões em emergência \newline

7) Informação sobre as pistas: se estão livres ou com qual avião (id) elas estão ocupadas.\newline

\begin{lstlisting}[label={list:first},caption= Parte do código em queue.h que imprime informações dos aviões (C++).]
    void Queue::printQueue()
    {
        string emergencia, pouso;
        Node * i;
        i = ini->next;
        cout << endl;
        if(n > 0)
        { 
            cout << "    |    ID    | DESTINO/ORIGEM | EMERGENCIA | TEMPO DE ESPERA | COMBUSTIVEL | POUSO/DECOLAGEM |" << endl;
            cout << "    +----------+----------------+------------+-----------------+-------------+-----------------+" << endl;
        }
        else cout << "     Essa fila nao possui avioes" << endl;
        while(i != NULL)
        {
            if(i->plane.emergency)
                emergencia = "SIM";
            else
                emergencia = "NAO";

            if(i->plane.landing)
                pouso = "POUSO";
            else
                pouso = "DECOLAGEM";
            
            cout << "    |  " << i->plane.id << " |      " << i->plane.from_to << "       |    " << emergencia << "     |        " << i->plane.waiting_time << "        |     ";
            if(i->plane.fuel >= 10) cout << i->plane.fuel << "      | " << pouso << endl;
            else cout << i->plane.fuel << "       | " << pouso << endl;
            i = i->next;
        }
        cout << endl;
    }   
\end{lstlisting}

Ou seja, cada fila é impressa em formato de tabela e contém as informações pedidas no enunciado do ep. \newline\newline

\section*{Exemplos}

Os exemplos foram feitos com base nos exemplos enviados pelos monitores no email. \newline

EXEMPLO 1:\newline

Entrada: \newline

\noindent\fbox{%
    \parbox{\textwidth}{%
    4
    4
    LA329 
    ACA
    1
    0
    4
    10
    LA563 
    ADZ
    1
    0
    4
    40
    LA923
    ANF
    1
    0
    4
    10
    LA734
    AQP
    1
    0
    4
    10
    0
    0
    0
    }%
}\newline
    
Saída:\newline

No arquivo out.txt\newline

EXEMPLO 2:\newline

Entrada:\newline

\noindent\fbox{%
    \parbox{\textwidth}{%
    1
    4
    LA329
    ACA
    1
    0
    2
    20
    LA563
    ADZ
    1
    0
    2
    20
    LA923
    ANF
    1
    0
    2
    12
    LA734
    AQP
    1
    0
    2
    32
    }%
}\newline

Saída:\newline

No arquivo out2.txt\newline

EXEMPLO 3:

Entrada:\newline

\noindent\fbox{%
    \parbox{\textwidth}{%
    19
    7
    LA329 
    GRU
    0
    0
    10
    20
    LA563
    GRU
    0
    0
    4
    20
    LA923
    GRU
    0
    0
    5
    20
    LA734
    GRU
    0
    0
    6
    40
    LA140
    GRU
    0
    0
    23
    60
    JB666
    GRU
    0
    1
    10
    30
    LA832
    BJX
    1
    0
    3
    10
    0
    0
    0
    0
    6
    LA485
    BPS
    1
    0
    10
    20
    LA300
    BHI
    1
    0
    10
    20
    LA887
    BVB
    1
    0
    10
    30
    LA993
    CAU
    1
    0
    10
    23
    LA554
    CBB
    1
    0
    10
    13
    LA111
    GRU
    0
    0
    10
    100
    0
    3
    LA344
    CGR
    1
    0
    5
    10
    LA461
    CKS
    1
    0
    9
    10
    LA875
    GRU
    0
    0
    20
    40
    0
    0
    0
    0
    0
    0
    0
    3
    LA673
    GRU
    0
    1
    30
    60
    LA899
    GRU
    0
    1
    30
    40
    LA505
    COR
    1
    0
    1
    20
    0
    0
    0
    }%
}\newline

Saída:\newline

No arquivo out3.txt\newline


OBS: Os arquivos de saída foram enviados no .tar junto com os outros arquivos.



\section*{Testes Aleatórios}

    Para gerar os testes aleatórios, criei um programa "rand" que está em rand.cpp que gera um arquivo texto (.txt) aleatóriamente no formato de entrada citado anteriormente. No gerador de aviões aleatórios (que está numa função, randPlane()) coloquei pra que a chance de um avião seja de emergência seja de 5/100.\newline

    O programa pede para digitar uma semente e depois digitar o nome do arquivo txt que será gerado. Depois disso, basta executar o programa mudando a entrada padrão para o txt que foi gerado (com "<" no terminal).

\begin{lstlisting}[label={list:first},caption= Parte do código em rand.cpp que gera aviões aleatórios.]
    Plane randPlane()
    {
        string r_id, r_from_to;
        bool r_landing;
        bool r_emergency;
        int r_fuel,r_estimated_time;
        int aux;
    
        aux = rand();
        aux = aux%30;
        r_id = "";
        r_id += comp[aux];
        for(int i = 0; i < 3; i++)
        {
            aux = rand();
            r_id += '0' + aux%10;
        } 
    
        aux = rand();
        aux = aux%30;
        r_from_to = places[aux];
    
    
        aux = rand();
        r_landing = aux%2;
    
        aux = rand();
        aux = aux%100;
        if(aux <= 5)
            r_emergency = 1;
        else
            r_emergency = 0;
    
        
        aux = rand();
        aux = aux%100;
        r_fuel = aux;
    
        aux = rand();
        aux = aux%100;
        r_estimated_time = aux;
    
        Plane aviao {r_id, r_from_to, r_landing, r_emergency, r_fuel, r_estimated_time};
        return aviao;
    } 
\end{lstlisting}


Partindo disso, fiz 5 testes que podem ser vistos na pasta do ep.\newline

Os teste foam feitos pras sementes 1, 2, 3, 4, 5. A respetivas entradas são rand1.txt, rand2.txt, rand3.txt, rand4.txt, rand5.txt. As respetivas saídas são outrand1.txt, outrand2.txt, outrand3.txt, outrand4.txt, outrand5.txt.\newline


\end{document}